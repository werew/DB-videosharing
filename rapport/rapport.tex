\documentclass[twoside,openright,a4paper,11pt,french]{article}
\usepackage[utf8]{inputenc}
\usepackage[french]{babel}
\usepackage[T1]{fontenc}
\usepackage{emptypage}
\usepackage{amsmath}

% Utilisation d'url
\usepackage{url}
\urlstyle{sf}

% Utilisation d'images, stockées dans le répertoire ./pics/
\usepackage{graphicx}
\graphicspath{pics/}

% Définition des marges
\usepackage{geometry}
\geometry{
  left=25mm,
  right=25mm,
  top=25mm,
  bottom=25mm,
  foot=15mm
}

\begin{document}

\pagestyle{plain}
\setlength{\parindent}{0pt}
% La page de garde
\include{page-garde}


% La table des matières
\parskip=0pt
\tableofcontents

\newpage

%Start content

\section{Fichiers rendu et usage}
\subsection{Contenu du rapport}
\subsection{Contenu de l'archive}
Apres avoir ouvert l'archive {\it coniglio\_luigi.tar.gz} vous
trouverez les fichiers suivants:

\begin{itemize}
\item Ce rapport
\item Les fichiers {\bf ent-rel.pdf} et {\bf base.pdf} contenant
      des images representant respectivement le schema entite-association
      et le modele relationnel de la base.
\item Le fichier {\bf tables.sql} qui contient le code SQL pour la
      creatione des toutes les tables de la base.
\item Dans le fichier {\bf queries.sql} vous trouverez les solutions
      au cinq requetes faisant l'object de la premiere partie du projet.
\item Dans le fichier {\bf proc-fun.sql} vous trouverez les procedures
      et les fonctions PL/SQL demande.
\item Le fichier {\bf triggers.sql} contiens les code des declancheurs qui
      etablissent les contraintes dynamiques sur la base.  Ce fichier est divise en
      deux partie: la premiere dedie aux contraintes dynamique explicitament
      demande dans la derniere partie du sujet, et une deuxieme partie contennent les
      contraintes dynamiques que j'ai cree conformenent a la description de la
      base.
\item Le fichier {\bf data.sql} contiens diverses donnees permettant de
      facilement remplir la base.
\item Les fichier {\bf droptables.sql}
\end{itemize}

\subsection{Usage}
Dans l'ar   


\section{Description des tables}

\section{Contraintes d'integrite}


%End content

\end{document}
