\documentclass[twoside,openright,a4paper,11pt,french]{article}
\usepackage[utf8]{inputenc}
\usepackage[french]{babel}
\usepackage[T1]{fontenc}
\usepackage{emptypage}
\usepackage{amsmath}

% Utilisation d'url
\usepackage{url}
\urlstyle{sf}

% Utilisation d'images, stockées dans le répertoire ./pics/
\usepackage{graphicx}
\graphicspath{pics/}

% Définition des marges
\usepackage{geometry}
\geometry{
  left=25mm,
  right=25mm,
  top=25mm,
  bottom=25mm,
  foot=15mm
}

\usepackage{listings}
\usepackage{color}

\definecolor{dkgreen}{rgb}{0,0.6,0}
\definecolor{gray}{rgb}{0.5,0.5,0.5}
\definecolor{mauve}{rgb}{0.58,0,0.82}

\lstset{frame=tb,
  language=SQL,
  aboveskip=3mm,
  belowskip=3mm,
  showstringspaces=false,
  columns=flexible,
  basicstyle={\small\ttfamily},
  numbers=none,
  numberstyle=\tiny\color{gray},
  keywordstyle=\color{blue},
  commentstyle=\color{dkgreen},
  stringstyle=\color{mauve},
  breaklines=true,
  breakatwhitespace=true,
  tabsize=3
}
\begin{document}

\pagestyle{plain}
\setlength{\parindent}{0pt}
% La page de garde
\include{page-garde}


% La table des matières
\parskip=0pt
\tableofcontents

\newpage

%Start content

\section{Fichiers rendu et usage}
\subsection{Contenu du rapport}
L'objectif de ce rapport est d'abord celui de illustrer la structure de la base
de donnees et en particulier les choix qui ont ete prises lors de
l'implementation. Pour accelerer/simplifier l'utilisation du travail rendu, la
partie initial de ce rapport decrit le contenu des fichiers et leurs usage.

\subsection{Contenu de l'archive}
Apres avoir ouvert l'archive {\it coniglio\_luigi.tar.gz} vous
trouverez les fichiers suivants:
\smallbreak
\begin{itemize}
\item Ce rapport
\item Les fichiers {\bf ent-rel.pdf} et {\bf base.pdf} contenant
      des images representant respectivement le schema entite-association
      et le modele relationnel de la base.
\item Le fichier {\bf tables.sql} qui contient le code SQL pour la
      creatione des toutes les tables de la base.
\item Dans le fichier {\bf queries.sql} vous trouverez les solutions
      au cinq requetes faisant l'object de la premiere partie du projet.
\item Dans le fichier {\bf proc-fun.sql} vous trouverez les procedures
      et les fonctions PL/SQL demande et des requet qui donnant des 
      examples d'utilisation.
\item Le fichier {\bf triggers.sql} contiens les code des declancheurs qui
      etablissent les contraintes dynamiques sur la base.  Ce fichier est divise en
      deux partie: la premiere dedie aux contraintes dynamique explicitament
      demande a la fin du sujet, et une deuxieme partie contennent les
      des autres contraintes dynamiques cree conformenent a la description de la
      base.
\item {\bf badqueries.sql} contiens des queries permettant de tester l'efficacite 
      des contraintes de la base.
\item Le fichier {\bf data.sql} permet de remplir la base.
\item Les fichier {\bf droptables.sql} peut etre utilise pour eliminer toutes 
      les tables de la base.
\item {\bf makedb.sql} est une raccourcie permettant de creer les tables, les
      declancheurs et remplire la base en une seule fois.
\end{itemize}

\bigbreak
La plupart du temps vous trouverez plusieurs solutions pour les requetes,
fonctions, procedures et les declancheurs (traduisent les contraintes) demandes
dans les sujet. Chaque version represente une implementation differente du
travail a effectuer, un autre moyen de resoudre le probleme  (p.ex. avec
curseur / sans curseur, avec des jointures / avec des sous requetes ...).
Chaque implementation est accompagnee par une petite descrition qui en
documente les caracteristiques.  Pour des raisons de clarte, les solutions
apres la premiere sont mises en commentaire. 

\subsection{Usage}
Utilisez les fichier {\bf makedb.sql} pour creer la base:
\begin{center}
\colorbox{gray}{\lstinline[basicstyle=\ttfamily\color{black}]|> @makedb|}
\end{center}
Une fois termine la creation de la base vous pouvez proceder a l'execution des
requetes
\colorbox{gray}{\lstinline[basicstyle=\ttfamily\color{black}]|> @queries|},
a la creation et execution des fonctions et procedures PL/SQL
\colorbox{gray}{\lstinline[basicstyle=\ttfamily\color{black}]|> @proc-fun|},
ou bien tester les contraintes d'integrite de la base
\colorbox{gray}{\lstinline[basicstyle=\ttfamily\color{black}]|> @badqueries|}.

\smallbreak
Dans chacun des ces cas vous serez presente avec un affichage permettant de
facilement reconnetre sur l'ecran la sortie de chaque requete:

\begin{lstlisting}
%TODO insert example
\end{lstlisting}

Apres avoir termine, utilisez le fichier {\bf droptables.sql} pour supprimer
la base de donnees:
\begin{center}
\colorbox{gray}{\lstinline[basicstyle=\ttfamily\color{black}]|> @droptables|}.
\end{center}

\section{Description des tables}
Etant donne que la structure de la base de donnees traduit directement la
description fourni dans le sujet, cette partie du rapport se limitera a fournir
quelque éclaircissements sur des parties laisse a la "libre interpretation" des
etudiants.

\subsection{Tables {\it WebUser} et {\it UserPass}}

Les password des utilisateurs ne sont pas stockees dans la meme table avec les
autres informations (table {\it WebUser}
\footnote{Le nom {\it WebUser} derive du fait que let mot {\it User} est reserve dans Oracle:
https://docs.oracle.com/cd/B19306\_01/em.102/b40103/app\_oracle\_reserved\_words.htm}.
Cet approche valorise l'aspect securite et permet par
example de differentier les droits sur les tables
\footnote{Une meilleure solutions pour stocker les mots de passe d'un
utilisateur consiste a utiliser un serveur different pas accessible de
l'exterieur et donc  moin subsceptible aux attaques}.
La table {\it UserPass} contient les hashes des passwords et leur valeur de
salage.  Ce systeme represente aujourd'hui le minimum en termes de securite
dans le stockage des mots de passe des utilisateurs.


\subsection{Types}
Les choix derrier certains champs suivent des raison bien precises:
\begin{description}
\item[Boolean] Un champ de type boolean est code comme etant un caracter (type CHAR
	\footnote{Le type CHAR a une taille statique}). Les valeurs possibles pour sont
	'Y' pour {\it vrai} et 'N' pour {\it fausse}.
\item[Email] L'espace allue pour un champ de type 
\item[Country] 
\end{description}


\section{Contraintes d'integrite}


%End content

\end{document}
