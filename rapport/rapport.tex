\documentclass[twoside,openright,a4paper,11pt,french]{article}
\usepackage[utf8]{inputenc}
\usepackage[french]{babel}
\usepackage[T1]{fontenc}
\usepackage{emptypage}
\usepackage{amsmath}

% Utilisation d'url
\usepackage{url}
\urlstyle{sf}

% Utilisation d'images, stockées dans le répertoire ./pics/
\usepackage{graphicx}
\graphicspath{pics/}

% Définition des marges
\usepackage{geometry}
\geometry{
  left=25mm,
  right=25mm,
  top=25mm,
  bottom=25mm,
  foot=15mm
}

\usepackage{listings}
\usepackage{color}

\definecolor{dkgreen}{rgb}{0,0.6,0}
\definecolor{gray}{rgb}{0.8,0.8,0.8}
\definecolor{mauve}{rgb}{0.58,0,0.82}

\lstset{frame=tb,
  language=SQL,
  aboveskip=3mm,
  belowskip=3mm,
  showstringspaces=false,
  columns=flexible,
  basicstyle={\small\ttfamily},
  numbers=none,
  numberstyle=\tiny\color{gray},
  keywordstyle=\color{blue},
  %commentstyle=\color{dkgreen},%
  stringstyle=\color{mauve},
  breaklines=true,
  breakatwhitespace=true,
  tabsize=3
}
\begin{document}

\pagestyle{plain}
\setlength{\parindent}{0pt}
% La page de garde
\include{page-garde}


% La table des matières
\parskip=0pt
\tableofcontents


\vspace{5cm}

%Start content

\section{Fichiers rendus et usage}
\subsection{Contenu du rapport}
L'objectif de ce rapport est d'abord celui d'illustrer la structure de la base
de données et en particulier les choix qui ont été prises lors de
l'implémentation. Pour accelerer/simplifier l'utilisation du travail rendu, la
partie initiale de ce rapport décrit le contenu des fichiers et leur usage.

\subsection{Contenu de l'archive}
Après avoir ouvert l'archive {\it coniglio\_luigi.tar.gz} vous
trouverez les fichiers suivants:
\smallbreak
\begin{itemize}
\item Ce rapport
\item Les fichiers {\bf ent-rel.pdf} et {\bf base.pdf} contenant des images
      représentant respectivement le schéma entité-association et le modèle
      relationnel de la base.
\item Le fichier {\bf tables.sql} qui contient le code SQL pour la création 
      de toutes les tables de la base.
\item Le fichier {\bf triggers.sql} contiens les codes des déclencheurs qui
      établissent les contraintes dynamiques sur la base. Ce fichier est divisé en
      deux parties: la première dédié aux contraintes dynamiques explicitement
      demande à la fin du sujet, et une deuxième partie contenant les dés autres
      contraints dynamiques créés conformément à la description de la base.
\item Le fichier {\bf data.sql} permet de remplir la base.
\item {\bf makedb.sql} est une raccourcie permettant de créer les tables, les 
      déclencheurs et remplir la base en une seule fois.
\item Dans le fichier {\bf queries.sql} vous trouverez les solutions aux 
      cinq requêtes faisant l'objet de la première partie du projet.
\item Dans le fichier {\bf proc-fun.sql} vous trouverez les procédures et 
      les fonctions PL/SQL demandées et des requetés qui donnent des exemples d'utilisation.
\item Le fichier {\bf test-triggers.sql} permet de tester tous les déclencheur.
\item Le fichier {\bf droptables.sql} peut-être utilise pour éliminer toutes 
      les tables de la base.
\item Le fichier {\bf out.txt} contient un example de la sortie genere par 
      les fichiers {\it queries.sql, proc-fun.sql et test-triggers.sql}. 
\end{itemize}

\bigbreak
La plupart du temps vous trouverez plusieurs solutions pour les requêtes,
fonctions, procédures et les déclencheurs (traduisent les contraintes) demandés
dans le sujet. Chaque version représente une implémentation différente du
travail à effectuer, un autre moyen de résoudre le problème (p .ex. avec
curseur / sans curseur, avec des jointures / avec des sous-requêtes ...).
Chaque implémentation est accompagnée par une petite description qui en
documente les caractéristiques. Pour des raisons de clarté, les solutions après
la première sont mises en commentaire.

\subsection{Usage}
Utilisez le fichier {\bf makedb.sql} pour créer la base:
\begin{center}
\colorbox{gray}{\lstinline[basicstyle=\ttfamily\color{black}]|> @makedb|}
\end{center}
Une fois termine la création de la base vous pouvez procéder à l'exécution des requêtes
\colorbox{gray}{\lstinline[basicstyle=\ttfamily\color{black}]|> @queries|},
procéder à la création et exécution des fonctions et procédures PL/SQL
\colorbox{gray}{\lstinline[basicstyle=\ttfamily\color{black}]|> @proc-fun|}.
ou tester les déclencheurs
\colorbox{gray}{\lstinline[basicstyle=\ttfamily\color{black}]|> @test-triggers|}.

\smallbreak
Dans chacun de ces cas un affichage sera présent permettant de facilement
reconnaître  sur l'écran la sortie de chaque requête:
\vspace{1cm}

\begin{lstlisting}
SQL> @queries

***** Requete N. 1 ******************************************************
* Nombre de visionnages de videos par categories de videos, pour les            *
* visionnages de moins de deux semaines.                                        *
*************************************************************************

Category                                    Views
-------------------------   --------------------------------
Entertainment                                1
Science                                      2
Cinema                                       1
\end{lstlisting}

\vspace{1cm}
Après avoir terminé, utilisez le fichier {\bf droptables.sql} pour supprimer la base de données:
\begin{center}
\colorbox{gray}{\lstinline[basicstyle=\ttfamily\color{black}]|> @droptables|}
\end{center}

\newpage 
\section{Description des tables}
Étant donné que la structure de la base de données reflète directement la
description fournie dans le sujet, cette partie du rapport se limitera à
fournir quelques éclaircissements sur quelque choix personnel.

\subsection{Noms et contenu des tables}
\begin{table}[h]
  \centering
% On paramètre ici le placement du texte dans les cases, en mode paragraphe de 5cm de large dans la case de gauche ("p{5cm}") et automatique avec un alignement à droite dans la case de droite ('r')
  \begin{tabular}{| p{5cm} | l |}
    \hline
    \textbf{Table} & \textbf{Contenu} \\
    \hline
    Video & Les videos disponibles\\
    \hline
    ArchivedVideo  & Les videos archivées \\
    \hline
    WebUser & Profils des utilisateurs \\
    \hline
    UserPass & Mots de passe des utilisateurs \\
    \hline
    Program & Les émissions \\
    \hline
    Category & Les catégories disponibles \\
    \hline
    UserView & Les visionnages \\
    \hline
    UserSelection & Video favoris \\
    \hline
    Subscription & Les abbonements des utilisateurs \\
    \hline
    Preference & Préférences par categorie \\
    \hline
    Diffusion & Diffusions des video \\
    \hline
  \end{tabular}
  \caption{Contenu des tables}
  \label{tab:tables}
\end{table}


\subsection{Table {\it UserPass}}
Les passwords des utilisateurs ne sont pas stockées dans la même table où ils
sont stockés les autres informations d'un utilisateur (table {\it WebUser}
\footnote{Le nom {\it WebUser} dérive du fait que let mot {\it User} est un mot reservé dans Oracle:
https://docs.oracle.com/cd/B19306\_01/em.102/b40103/app\_oracle\_reserved\_words.htm}.
Cet approche valorise l'aspect securité et permet par
example de differentier les droits sur les tables
\footnote{Une meilleure solutions pour stocker les mots de passe d'un
utilisateur consiste à utiliser un serveur different pas accessible depuis
l'exterieur et donc moin subsceptible aux attaques}.
La table {\it UserPass} contient les hashes des passwords et leur valeur de
salage.  Ce systeme represente aujourd'hui le minimum en termes de securité
dans le stockage des mots de passe des utilisateurs.


\subsection{Types}
Certaines choix d'implementation relatives à certaines champs utilisés dans la
base meritent une petite decription:


\begin{description}
\item[Boolean] - 
Oracle n'implémente pas le type boolean, pour cette raison il a fallu utiliser
une autre stratégie pour les champs ayant une valeur de ce type.  Dans ce
projet le type boolean est replacé par un caractère (type CHAR) ayant comme
valeurs possibles 'Y' pour {\it vrai} et 'N' pour {\it faux}.

\item[Email] - En suivant la spécification du protocole SMTP
\footnote{RFC 5321 - https://tools.ietf.org/html/rfc5321.html}
la taille limite pour un champ contenant une email est de 320 caractères: 64
caractères pour le nom de l'utilisateur + @ + 255 caractères pour le nom du
domaine \footnote{ Une possibilité, pas implémente dans le travail rendu pour
des raisons de simplicité, était celle de stocker les noms de domaines dans une
table sépare pour éviter les redondances.}.

\item[Country] - En suivant les ses codes spécifiés dans la norme ISO 3166-1
alpha-2 il est possible d'utiliser seulement deux caractères pour stocker le
nom d'un pays: c'est pour cette raison que le type de ce champ est CHAR(2).

\end{description}

\newpage
\section{Contraintes d'Intégrité} En complément aux contraintes d'Intégrité
spécifie dans la dernière partie du sujet, les contraintes d'Intégrité
suivantes ont été implémenté dans la base.

\begin{itemize}
\item Un utilisateur ne peut pas visionner une vidéo qui n'a pas été diffusé,
par contre il peut le sélectionner (le mettre dans les favoris) par exemple
pour un visionnage futur. Trigger {\it ValidView}.

\item Un utilisateur ne peut pas visionner une vidéo qui a dépassé la date de
validité (même s'il n'a pas encore été archivée). Trigger {\it ValidView}.

\item Si la date de disponibilité n'est pas passée, une vidéo ne peut pas être supprimée. 
      Trigger {\it WaitExpiration }.

\item Comme spécifie le sujet: {\it "Après une diffusion, une vidéo sera
accessible sur le site en replay pendant au moins 7 jours"}. 
\footnote{NB: la deuxième contrainte demande dans le sujet ({\it"Si une diffusion d’une émission est
ajoutée, les dates de disponibilités seront mises à jour.  La nouvelle date de
fin de disponibilité sera la date de la dernière diffusion plus 4 jours."})
n'implique pas que cette contrainte soit satisfaite (p.ex. il ne couvre pas les
opérations d'UPDATE sur la table {\it Video}).} 
Trigger {\it BadExpiration}.

\item La date (année) de première diffusion d'une vidéo ne doit pas être
supérieure la date de la première diffusion dans la table {\it Diffusion}.
Triggers {\it FirstDiffusionCheck} et {\it VideoWasDiffused}.

\item  Au moment de sa création, le champ Time d'un visionnage ne peut pas être
supérieure à la date actuelle (SYSDATE): on interdit les visionnages "dans le
futur".
\footnote{C'est le cas typique d'une contrainte qui pourrait être remplacé par
l'intervention de l'application web (côté serveur) dans le choix de la date du
visionnage.}
Trigger {\it ValidView}.



\end{itemize}

%End content

\end{document}
