\documentclass[twoside,openright,a4paper,11pt,french]{article}
\usepackage[utf8]{inputenc}
\usepackage[french]{babel}
\usepackage[T1]{fontenc}
\usepackage{emptypage}
\usepackage{amsmath}

% Utilisation d'url
\usepackage{url}
\urlstyle{sf}

% Utilisation d'images, stockées dans le répertoire ./pics/
\usepackage{graphicx}
\graphicspath{pics/}

% Définition des marges
\usepackage{geometry}
\geometry{
  left=25mm,
  right=25mm,
  top=25mm,
  bottom=25mm,
  foot=15mm
}

\usepackage{listings}
\usepackage{color}

\definecolor{dkgreen}{rgb}{0,0.6,0}
\definecolor{gray}{rgb}{0.8,0.8,0.8}
\definecolor{mauve}{rgb}{0.58,0,0.82}

\lstset{frame=tb,
  language=SQL,
  aboveskip=3mm,
  belowskip=3mm,
  showstringspaces=false,
  columns=flexible,
  basicstyle={\small\ttfamily},
  numbers=none,
  numberstyle=\tiny\color{gray},
  keywordstyle=\color{blue},
  %commentstyle=\color{dkgreen},%
  stringstyle=\color{mauve},
  breaklines=true,
  breakatwhitespace=true,
  tabsize=3
}
\begin{document}

\pagestyle{plain}
\setlength{\parindent}{0pt}
% La page de garde
\include{page-garde}


% La table des matières
\parskip=0pt
\tableofcontents


\vspace{5cm}

%Start content

\section{Fichiers rendu et usage}
\subsection{Contenu du rapport}
L'objectif de ce rapport est d'abord celui de illustrer la structure de la base
de donnees et en particulier les choix qui ont ete prises lors de
l'implementation. Pour accelerer/simplifier l'utilisation du travail rendu, la
partie initial de ce rapport decrit le contenu des fichiers et leurs usage.

\subsection{Contenu de l'archive}
Apres avoir ouvert l'archive {\it coniglio\_luigi.tar.gz} vous
trouverez les fichiers suivants:
\smallbreak
\begin{itemize}
\item Ce rapport
\item Les fichiers {\bf ent-rel.pdf} et {\bf base.pdf} contenant
      des images representant respectivement le schema entite-association
      et le modele relationnel de la base.
\item Le fichier {\bf tables.sql} qui contient le code SQL pour la
      creatione des toutes les tables de la base.
\item Dans le fichier {\bf queries.sql} vous trouverez les solutions
      au cinq requetes faisant l'object de la premiere partie du projet.
\item Dans le fichier {\bf proc-fun.sql} vous trouverez les procedures
      et les fonctions PL/SQL demande et des requet qui donnant des 
      examples d'utilisation.
\item Le fichier {\bf triggers.sql} contiens les code des declancheurs qui
      etablissent les contraintes dynamiques sur la base.  Ce fichier est divise en
      deux partie: la premiere dedie aux contraintes dynamique explicitament
      demande a la fin du sujet, et une deuxieme partie contennent les
      des autres contraintes dynamiques cree conformenent a la description de la
      base.
\item Le fichier {\bf data.sql} permet de remplir la base.
\item Les fichier {\bf droptables.sql} peut etre utilise pour eliminer toutes 
      les tables de la base.
\item {\bf makedb.sql} est une raccourcie permettant de creer les tables, les
      declancheurs et remplire la base en une seule fois.
\item Le fichier {\bf out.txt} contient un example de la sortie genere par 
      les fichiers {\it makedb.sql, queries.sql et proc-fun.sql}. {\bf ATTENTION} le 
      contenu de ce fichier a ete genere Jeudi 3 Novembre 2016 sur le serveur Oracle
      disponible sur la machine {\it osr-etudiant.unistra.fr}: le produit
      d'une execution dans une autre date pourrait sans doute etre different du 
      contenu de {\it out.txt}
      \footnote{Pour visualiser ce fichier je vous conseille de ne pas utiliser
{\it vim} qui dans ce cas particulier pose quelque probleme d'affichage.
Utilisez un autre editeur de texte (gedit par example) ou un affichage a
l'ecran en utilisant la commande {\it cat}}.
\end{itemize}

\bigbreak
La plupart du temps vous trouverez plusieurs solutions pour les requetes,
fonctions, procedures et les declancheurs (traduisent les contraintes) demandes
dans les sujet. Chaque version represente une implementation differente du
travail a effectuer, un autre moyen de resoudre le probleme  (p.ex. avec
curseur / sans curseur, avec des jointures / avec des sous requetes ...).
Chaque implementation est accompagnee par une petite descrition qui en
documente les caracteristiques.  Pour des raisons de clarte a l'execution, les
solutions apres la premiere sont mises en commentaire. 

\subsection{Usage}
Utilisez les fichier {\bf makedb.sql} pour creer la base:
\begin{center}
\colorbox{gray}{\lstinline[basicstyle=\ttfamily\color{black}]|> @makedb|}
\end{center}
Une fois termine la creation de la base vous pouvez proceder a l'execution des
requetes
\colorbox{gray}{\lstinline[basicstyle=\ttfamily\color{black}]|> @queries|},
ou a la creation et execution des fonctions et procedures PL/SQL
\colorbox{gray}{\lstinline[basicstyle=\ttfamily\color{black}]|> @proc-fun|}.

\smallbreak
Dans chacun des ces cas vous serez presente avec un affichage permettant de
facilement reconnetre sur l'ecran la sortie de chaque requete:

\begin{lstlisting}
SQL> @queries

***** Requete N. 1 ******************************************************
* Nombre de visionnages de videos par categories de videos, pour les            *
* visionnages de moins de deux semaines.                                        *
*************************************************************************

Category                                    Views
-------------------------   --------------------------------
Entertainment                                1
Science                                      2
Cinema                                       1
\end{lstlisting}

Apres avoir termine, utilisez le fichier {\bf droptables.sql} pour supprimer
la base de donnees:
\begin{center}
\colorbox{gray}{\lstinline[basicstyle=\ttfamily\color{black}]|> @droptables|}.
\end{center}

\section{Description des tables}
Etant donne que la structure de la base de donnees traduit directement la
description fourni dans le sujet, cette partie du rapport se limitera a fournir
quelque éclaircissements sur des parties laisse a la "libre interpretation" des
etudiants.

\newpage
\subsection{Noms et contenu des tables}
\begin{table}[h]
  \centering
% On paramètre ici le placement du texte dans les cases, en mode paragraphe de 5cm de large dans la case de gauche ("p{5cm}") et automatique avec un alignement à droite dans la case de droite ('r')
  \begin{tabular}{| p{5cm} | l |}
    \hline
    \textbf{Table} & \textbf{Contenu} \\
    \hline
    Video & Les videos disponibles\\
    \hline
    ArchivedVideo  & Les videos archivee (non plus disponibles)\\
    \hline
    WebUser & Profils des utilisateurs \\
    \hline
    UserPass & Pot de passe des utilisateurs \\
    \hline
    Program & Les emissions \\
    \hline
    Category & Les categories disponibles \\
    \hline
    UserView & Les visionnages \\
    \hline
    UserSelection & Video favoris \\
    \hline
    Subscription & Les abbonements des utilisateurs \\
    \hline
    Preference & Preferences par categorie \\
    \hline
    Diffusion & Diffusions des video \\
    \hline
  \end{tabular}
  \caption{Contenu des tables}
  \label{tab:tables}
\end{table}


\subsection{Table {\it UserPass}}

Les password des utilisateurs ne sont pas stockees dans la meme table avec les
autres informations (table {\it WebUser}
\footnote{Le nom {\it WebUser} derive du fait que let mot {\it User} est un mot reserve dans Oracle:
https://docs.oracle.com/cd/B19306\_01/em.102/b40103/app\_oracle\_reserved\_words.htm}.
Cet approche valorise l'aspect securite et permet par
example de differentier les droits sur les tables
\footnote{Une meilleure solutions pour stocker les mots de passe d'un
utilisateur consiste a utiliser un serveur different pas accessible de
l'exterieur et donc  moin subsceptible aux attaques}.
La table {\it UserPass} contient les hashes des passwords et leur valeur de
salage.  Ce systeme represente aujourd'hui le minimum en termes de securite
dans le stockage des mots de passe des utilisateurs.


\subsection{Types}
Certaines choix d'implementation relatives aux certaines champs utilise dans la
base meritent une petite decription:


\begin{description}
\item[Boolean] - Oracle n'implemente pas le type boolean, pour cette raison il a
fallu utiliser une autre strategie pour les champs ayant un valeur de ce type.
Dans ce projet le type boolean est replace par un caracter (type CHAR) ayant
comme valeurs possibles 'Y' pour {\it vrai} et 'N' pour {\it fausse}.

\item[Email] - En suivant la specification du protocole SMTP
\footnote{RFC 5321 - https://tools.ietf.org/html/rfc5321.html}
la taille limite pour un champ contanant une email est de 320 caracteres: 64
caracteres pour le nom de l'utilisateur + @ + 255 caracteres pour le nom du
domaine\footnote{Une possibilite, pas implemente dans le travail rendu pour des
raisons de simplicite , etait celle de stocker les noms de domaines dans une
table separe pour eviter les redondances.}.

\item[Country] - En suivant les ses codes specifiees dans la norme ISO 3166-1
alpha-2 il est possible de utiliser seulement deux caracteres pour stocker le
nom d'un pays: c'est pour cette raison que le type de cet champ est CHAR(2).

\end{description}

\newpage
\section{Contraintes d'integrite}
En complement aux contraintes d'integrite specifie dans la derniere partie du
sujet, les contraintes d'integrite suivants ont ete implemente dans la base:

\begin{itemize}
\item Un utilisateur ne peut pas visionner un video qui n'a pas ete diffuse,
par contre il peut le selectionner (le mettre dans les favoris) par example
pour un visionnage future.

\item Un utilisateur ne peut pas visionner un video qui a depasse la date de validite
(meme si il n'a pas encore ete archivee).

\item Si la date de disponibilite n'est pas passe, un video ne peut pas etre supprime

\item Comme specifie dans le sujet: {\it "Après une diffusion, une vidéo sera
accessible sur le site en replay pendant au moins 7 jours"}.
\footnote{NB: le deuxieme contrainte demande dans le sujet ({\it"Si une diffusion d’une émission est
ajoutée, les dates de disponibilités seront mises à jour.  La nouvelle date de
fin de disponibilité sera la date de la dernière diffusion plus 4 jours."}) n'implique
pas que ce contrainte soit satisfee (p.ex. il ne couvre pas les operation de UPDATE 
sur la table {\it Video}).}

\item La date (annee) de premiere diffusion d'un video ne doit pas
etre superieure la date de la premiere diffusion dans la table {\it Diffusion}.

\item Au moment de sa creation, le champ Time d'un visionnage ne peut pas etre superieure
a la date actuelle (SYSDATE): on interdit les visionnages "dans le futur".
\footnote{C'est le cas typique d'un contrainte qui pourrait etre ramplace par l'intervention de
l'application web (cote serveur) dans la choix de la date du visionnage.}

\end{itemize}

%End content

\end{document}
