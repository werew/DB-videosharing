\documentclass[twoside,openright,a4paper,11pt,french]{article}
\usepackage[utf8]{inputenc}
\usepackage[french]{babel}
\usepackage[T1]{fontenc}
\usepackage{emptypage}
\usepackage{amsmath}

% Utilisation d'url
\usepackage{url}
\urlstyle{sf}

% Utilisation d'images, stockées dans le répertoire ./pics/
\usepackage{graphicx}
\graphicspath{pics/}

% Définition des marges
\usepackage{geometry}
\geometry{
  left=25mm,
  right=25mm,
  top=25mm,
  bottom=25mm,
  foot=15mm
}

\usepackage{listings}
\usepackage{color}

\definecolor{dkgreen}{rgb}{0,0.6,0}
\definecolor{gray}{rgb}{0.5,0.5,0.5}
\definecolor{mauve}{rgb}{0.58,0,0.82}

\lstset{frame=tb,
  language=SQL,
  aboveskip=3mm,
  belowskip=3mm,
  showstringspaces=false,
  columns=flexible,
  basicstyle={\small\ttfamily},
  numbers=none,
  numberstyle=\tiny\color{gray},
  keywordstyle=\color{blue},
  commentstyle=\color{dkgreen},
  stringstyle=\color{mauve},
  breaklines=true,
  breakatwhitespace=true,
  tabsize=3
}
\begin{document}

\pagestyle{plain}
\setlength{\parindent}{0pt}
% La page de garde
\include{page-garde}


% La table des matières
\parskip=0pt
\tableofcontents

\newpage

%Start content

\section{Fichiers rendu et usage}
\subsection{Contenu du rapport}
\subsection{Contenu de l'archive}
Apres avoir ouvert l'archive {\it coniglio\_luigi.tar.gz} vous
trouverez les fichiers suivants:

\begin{itemize}
\item Ce rapport
\item Les fichiers {\bf ent-rel.pdf} et {\bf base.pdf} contenant
      des images representant respectivement le schema entite-association
      et le modele relationnel de la base.
\item Le fichier {\bf tables.sql} qui contient le code SQL pour la
      creatione des toutes les tables de la base.
\item Dans le fichier {\bf queries.sql} vous trouverez les solutions
      au cinq requetes faisant l'object de la premiere partie du projet.
\item Dans le fichier {\bf proc-fun.sql} vous trouverez les procedures
      et les fonctions PL/SQL demande et des requet qui donnant des 
      examples d'utilisation.
\item Le fichier {\bf triggers.sql} contiens les code des declancheurs qui
      etablissent les contraintes dynamiques sur la base.  Ce fichier est divise en
      deux partie: la premiere dedie aux contraintes dynamique explicitament
      demande a la fin du sujet, et une deuxieme partie contennent les
      des autres contraintes dynamiques cree conformenent a la description de la
      base.
\item {\bf badqueries.sql} contiens des queries permettant de tester l'efficacite 
      des contraintes de la base.
\item Le fichier {\bf data.sql} permet de remplir la base.
\item Les fichier {\bf droptables.sql} peut etre utilise pour eliminer toutes 
      les tables de la base.
\item {\bf makedb.sql} est une raccourcie permettant de creer les tables, les
      declancheurs et remplire la base en une seule fois.
\end{itemize}

\subsection{Usage}
Utilisez les fichier {\bf makedb.sql} pour creer la base:
\begin{center}
\colorbox{gray}{\lstinline[basicstyle=\ttfamily\color{black}]|> @makedb|}
\end{center}
Une fois termine la creation de la base vous pouvez proceder a l'execution des
requetes
\colorbox{gray}{\lstinline[basicstyle=\ttfamily\color{black}]|> @queries|},
a la creation et execution des fonctions et procedures PL/SQL
\colorbox{gray}{\lstinline[basicstyle=\ttfamily\color{black}]|> @proc-fun|},
ou bien tester les contraintes d'integrite de la base
\colorbox{gray}{\lstinline[basicstyle=\ttfamily\color{black}]|> @badqueries|}.

\smallbreak
Dans chacun des ces cas vous serez presente avec un affichage permettant de
reconnetre sur l'ecran la sortie de chaque requete:
\begin{lstlisting}
%TODO insert example

\end{lstlisting}


\section{Description des tables}

\section{Contraintes d'integrite}


%End content

\end{document}
